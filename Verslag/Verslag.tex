\documentclass[a4paper,openany]{uantwerpenassignment}

\usepackage[dutch]{babel}
\usepackage{titlesec}

\facultyacronym{TI}

\title{Chess AI-Agent}
\subtitle{5-Artificiële Intelligentie}
\author{Mathias Maes, Tijs Van Alphen \\en Willem Van der Elst}

\programme{BA}{IW}{EI}

\academicyear{2020-2021}

\publisher{}

\titleformat{\chapter}{\normalfont\huge\bfseries}{\thechapter.}{10pt}{\huge\bfseries}

\begin{document}

\maketitle

\tableofcontents

\chapter{Keuze}

Om onze keuze te maken hadden we een lijst opgesteld met positieve en negatieve punten opgesteld.

[VOEG + en - punten toe]

Uit deze lijst hadden we 2 keuzes die voor ons er ver bovenuit staken Q-Learning en Minimax.

Uiteindelijk ging de keuze naar beide agents. We hadden namelijk een thesis\cite{rl} gevonden die een minimax agent gebruikte als feature voor een generalized Q-Learner.

\chapter{Q-Learning Agent}

\section{Generalization}

De eerste agent die we moesten maken was de Q-Learner deze is generalized omdat zoals bij de minpunten stonde van general search algoritmes. 

\section{Features}

\chapter{Alpha-Beta Pruning}

\section{Utility}

\chapter{It's morphing time}

\bibliography{sources} 
\bibliographystyle{ieeetr}

\end{document}
